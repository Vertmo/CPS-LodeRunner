\documentclass{article}
\usepackage[a4paper,margin=1in]{geometry}
\usepackage{listings}

\title{CPS - Projet - LodeRunner\\Rapport}
\author{Basile Pesin, David Sreng\\Sorbonne Université}

\begin{document}
\maketitle

\section{Manuel d'utilisation}

Le projet a été fait en utilisant Java 8. Un fichier build.xml permet de compiler le projet, d'exécuter le jeu, d'ouvrir l'éditeur de niveau et de lancer les test.\\
Les différentes commandes sont:
\begin{itemize}
\item \textbf{ant compile} pour compiler (mais c'est aussi une dépendance des autres commandes, il n'est donc pas nécessaire de l'exécuter explicitement)
\item \textbf{ant run} pour lancer le jeu
\item \textbf{ant run-editor} pour lancer l'éditeur
\item \textbf{ant test} pour lancer les tests MBT (et \textbf{ant test-bug} pour lancer les tests sur les implémentations buggées). Les rapports de tests respectifs sont alors générés dans les fichiers \textit{report.txt} et \textit{report-bug.txt}
\end{itemize}
Les commandes dans le jeu sont:
\begin{itemize}
\item \textbf{Mouvements}: touches fléchées
\item \textbf{DigL (et ouvrir une porte à gauche)}: E, \textbf{DigR (et ouvrir une porte à droite)}: R
\item \textbf{ShootL}: D, \textbf{ShootR}: F
\end{itemize}
L'éditeur intègre un interface graphique avec une toolbar présentant les différents éléments possible ajoutables dans le niveau. La nature des cellules fonctionne comme on pourrait s'y attendre. Pour les contenus des cellules, on peut enlever un contenu en cliquant dessus si c'est le contenu choisi dans la toolbar (par exemple pour enlever un garde, il faut choisir "garde" dans la toolbar puis cliquer sur la cellule contenant le garde). 

\section{Travail réalisé}

\subsection{Quelques détails}

Nous avons terminé le projet de base avec les implémentations, les contrats et les tests MBT. Nous avons tenté de suivre au plus près possible le comportement du jeu original, parfois en nous éloignant de la spécification du projet (en particulier en ce qui concerne les interactions avec différents types de cellules). De plus, noua avons ajouté trois autres extensions (qui elles s'éloignent parfois du jeu original) qui seront détaillées après. Le jeu et l'éditeur intègrent une interface graphique utilisant JavaFX (qui est normalement livré avec toute installation complète de Java 8).\\

En tout, on a écrit 166 tests. Comme la distribution des tests présente dans \textit{test.pdf} l'indique, on s'est principalement focalisé sur les tests de transition et de préconditions, où on atteint des couvertures proches des 100\%. En revanche, on a implémenté moins de tests de paires de transitions (même si certains tests de transitions recouvrent des tests de paires de transitions) principalement pour des raisons de temps. On a aussi implémenté au moins un test de scénario par service (sauf pour le service \textrm{Screen}, pour lequel cela n'avait pas beaucoup de sens).

\subsection{Difficultés rencontrés}

Pour toute la partie commune du projet, nous n'avons pas réellement rencontré de difficultés, en dehors de quelques imprécisions dans le sujet fourni qu'il nous a fallu corriger dans notre spécification. Pour les implémentations, il suffisait simplement de suivre les spécifications définies. Pour les contrats, le seul point à noter est l'utilisation de \textit{clone}, \textit{equals} ou \textit{hashcode} dans certaines situation afin de vérifier des post conditions. Comme nous avons utilisé la classe \textit{HashSet<T>} de Java pour gérer les collections vues comme des ensembles dans la spécification, il a parfois était difficile de définir des ces méthodes de manière cohérente, afin que les comparaisons des contrats fonctionnent correctement, en particulier pour le service \textrm{Guard}. Enfin pour les tests MBT, l'utilisation des contrats permet rapidement de vérifier le bon fonctionnement des implémentations (même si nous avons aussi ajouté quelques \textit{assertEquals} pour vérifier les points importants des oracles par nous même).\\

Ce n'est qu'à partir l'ajout des extensions que certains problèmes sont apparus. En effet, le projet de base avait une spécification prenant en compte la globalité des services existants. En ajoutant des extensions, il a fallu modifier la spécification globale à certains endroit ce qui a provoqué des erreurs dans les contrats et donc des erreurs dans les tests MBT. Nous avons donc dû passer du temps à refaire fonctionner tous les contrats et les tests MBT.

\section{Extensions}
La spécification des extensions décrites ci-dessous est disponible dans le fichier \textit{spec.pdf}, dans les sections du même nom.

\subsection{Améliorations des gardes}
Dans la spécification actuelle du jeu, les gardes ne sont pas très astucieux: ils refusent parfois de prendre une route allant en ligne droite jusqu'au joueur, et de faire certains mouvements assez simples. Sans aller jusqu'à programmer un algorithme de path-finding pour les gardes, on va améliorer quelques peu ces mouvements simples.

\subsubsection{Descendre les échelles, tomber des rails}
Deux des positions les plus facilement repérable dans lesquels les gardes sont bloqués est au sommet d'une échelle, ou accroché à un rail: en effet, dans la spécification actuelle, dans aucun de ces deux cas le comportement du garde n'est \textbf{Down} si le joueur se trouve en dessous de lui. On peut donc ajouter ces cas à la disjonction pour le \textbf{Down} dans les invariants du garde. On remarque que ces invariants sont déjà assez complexes, et l'ajout de ces cas spéciaux les rends de plus en plus difficiles à lire.

\subsubsection{Quitter les échelles}
Dernière faiblesse de nos gardes: ils ne savent pas quitter les échelles autrement qu'en ayant ``quelque chose sous les pieds'', que ce soit une case non-vide, ou un autre garde, alors que rien ne les empêcherai physiquement de quitter l'échelle si ils le souhaitaient. On décide donc de modifier l'IA des gardes pour qu'ils puissent se déplacer latéralement depuis une échelle, dans le cas ou ce déplacement permet de s'accrocher à un rail, ou de prendre pied sur une plateforme ou du métal. Encore une fois, cette modification s'applique dans les invariants en liens avec l'observateur \textrm{Behaviour}, et complique encore leur définitions en les rendants plus illisibles. De plus, il devient de plus en plus difficile de couvrir tous les sous-cas possibles pour un garde. On décide donc d'arrèter de modifier l'IA des gardes (de plus grosses modifications, comme un algorithme de pathfinding demanderaient sans doute une réécriture complète des invariants de \textrm{Behaviour}, ce pourquoi on a malheureusement pas le temps).

\subsubsection{Un pas sur deux}
Après toutes ces améliorations faites aux gardes, ceux-ci deviennent de plus en plus dangereux pour le joueur. Pour contrebalancer ce danger, on décide d'affaiblir considérablement les gardes en divisant par deux leur capacité d'action: autrement dit, les gardes n'agissent plus qu'un \textrm{step} de l'\textrm{Engine} sur deux, plutôt que tous les \textrm{step}. Pour cela, on doit modifier le service \textrm{Engine} plutôt que le service \textrm{Guard}, en introduisant un nouvel observateur \textrm{GuardTurn: [Engine] $\rightarrow$ boolean} qui s'inverse à chaque \textrm{step}. Le \textrm{Guard::step} n'est alors déclenché que si cet observateur est vrai. On observe que modifier l'appel des \textrm{step} de cette façon modifie double aussi le temps potentiel passé dans un trou par le garde (puisque \textrm{Guard::TimeInHole} augmente deux fois moins vite), et donc diminue les chances du garde de se sortir du trou. On pourrait si on voulait tout simplement diviser par 2 le seuil au delà duquel le garde sort du trou, mais cet équilibrage nous paraît correct.

\subsection{Plus de cellules}
Dans cette extension, on décide de complexifier le gameplay de notre jeu en ajoutant de nouvelles variantes de cellules permettant de créer des puzzle plus complexes.

\subsubsection{Pièges}
La première case qu'on ajoute est assez simple: il s'agit d'un piège, c'est à dire une case \textbf{TRP} ressemblant au premier abord à une plateforme normale, mais se dérobant dès que le joueur (ou un garde) marche dessus.
L'implémentation du piège nécessite principalement des modifications dans le service \textrm{Screen}, ou l'on ajoute un nouvel opérateur \textrm{TriggerTrap: [Screen] $\times$ int $\times$ int $\rightarrow$ [Screen]} déclenchant un des pièges, c'est à dire transformant la case \textbf{TRP} en case \textbf{EMP}. De plus, il faut aussi effectuer des modifications au niveau du service \textrm{Engine}, principalement dans les post-conditions de \textrm{Step}, pour que le passage du joueur ou d'un garde au dessus de la case du piège déclenche effectivement celui-ci. Il est intéressant de remarquer que, pour conserver la cohérence de la spécification, le piège se déclenche après le \textrm{Step} du joueur ou garde qui l'a déclenché (sinon il y aurait une confusion sur le contenu de la case sous les pieds du personnage à l'instant du déclenchement).\\

On note également des modifications dans les services \textrm{Character} et \textrm{Guard}, pour que les pièges soient considérés comme des plateformes avant d'être déclenchés (entre autres pour qu'ils bloquent un mouvement latéral). Ces changements étant peu intéressants d'un point de vue de la spécification, on n'a pas pris la peine de recopier les nouvelles spécifications de ces services dans le fichier \textit{spec.pdf}.

\subsubsection{Portails}
Le deuxième type de case qu'on ajoute n'est en fait pas (pour des raisons de spécifications sur lesquelles on reviendra) en fait pas vraiment un type de case. Il s'agit d'un système de portails unidirectionnels (certains portails bleus sont des portails d'entrées, d'autres oranges sont des portails de sorties) que le joueur peut traverser. Les portails fonctionnent par paires, de sortes qu'entrer dans un portail bleu donné amènera toujours au même portail orange correspondant. Cependant, la correspondance des paires de portails n'est pas clairement marquée sur l'écran de jeu, de sorte que le joueur ait à deviner (ou à se souvenir) du portail correspondant à celui ou il veut se rendre, à la façon d'un jeu de memory.\\

Comme les cases sont implémentées par une énumération (et que nous ne souhaitions pas modifier cette implémentation, puisqu'il aurait alors fallu réécrire une bonne partie de la spécification et de l'implémentation), on ne pouvait pas stocker dans une case l'information des coordonnées du portail de sortie. On a donc décidé de stocker les paires de portails comme une information supplémentaire pour l'\textrm{Engine}. On peut donc un accéder avec l'observateur \textrm{Portals: [Engine] $\rightarrow$ PortalPair} (le type de données \textrm{PortalPair} contient tout simplement les coordonnées du portail d'entrée accessible avec \textrm{CoordPIn} et celles du portail de sortie avec \textrm{CoordPOut}.\\
On doit également modifier le constructeur \textrm{init} de \textrm{Engine} pour lui passer en plus des paramètre précédant un ensemble de \textrm{PortalPair}. Cela nous oblige à modifier pas mal du code utilisant ce \textrm{init} (on aurait aussi pu introduire un nouvel \textrm{init}) plutôt que de modifier l'existant, mais cela aurait compliqué la spécification).\\
Enfin, on ajoute au \textrm{Player} un opérateur \textrm{Teleport: [Player] $\times$ int $\times$ int $\rightarrow$ [Player]} qui comme son nom l'indique téléporte le joueur, c'est à dire le déplace instantanément aux coordonnées spécifiées. Cet opérateur est appelé par \textrm{Engine} à la place de \textrm{Player::step} quand le joueur se trouve sur une case contenant une entrée de portail.

\subsubsection{Portes et clés}
Enfin, le troisième et dernier type de cases qu'on introduit est la porte fermée à clef. Ces portes bloquent le passage à la fois du joueur et des gardes, et ne peuvent être ouvertes qu'au moyen d'une clef récupérée dans le niveau. Les clefs peuvent être utilisées sur n'importe quelle porte, mais sont à usage unique (à la manière des clefs d'un donjon de \textit{The Legend Of Zelda}). De plus, d'éventuelles clefs supplémentaires ne sont pas conservées d'un niveau au suivant (cela nuirait à l'indépendance des niveaux).\\

Les clés sont principalement gérées par le service \textrm{Player} grâce au nouvel observateur \textrm{NbKeys} (qui commence à 0 à l'initialisation du \textrm{Player}) et à l'opérateur \textrm{GrabKey} qui ajoute une clé au joueur. Ce dernier opérateur est appelé automatiquement par l'\textrm{Engine} quand le joueur se trouve sur la même case qu'un \textrm{Item} de type \textbf{Key} (la collecte des clés fonctionne donc à peu près comme la collecte de trésor). On ajoute que les gardes ne peuvent pas ramasser des clés, contrairement aux trésors.\\

Le joueur peut ensuite utiliser une clé pour ouvrir une porte: quand il se trouve directement à droite (respectivement à gauche) d'une porte fermée, le joueur peut invoquer la commande \textbf{DigL} (respectivement \textbf{DigR} pour ouvrir la porte (ce qui revient donc à presser la touche \textbf{q} ou \textbf{e}). On a décidé de réutiliser ces commandes plutôt que d'en introduire de nouvelles pour conserver la correspondance entre les touches du clavier et les commandes, et puisqu'il n'exists pas de cas où il y'aurait confusion entre l'action de creuser et celle d'ouvrir une porte (puisqu'on ne peut de toute façon pas creuser une case située sous une case non-vide comme une porte). Une fois ouverte, la case contenant précédemment la porte devient une case \textbf{EMP}.

\subsection{Pistolet}

Cette extension permet au Player de récupérer un pistolet dans le niveau afin de pouvoir tuer les gardes. Cela permet de donner un moyen de récupérer les trésors des gardes sans avoir à creuser un trou , ce qui est très utile dans un niveau avec seulement des platformes métaliques par exemple.\\

Le pistolet est un \textbf{Item} dont la nature est \textbf{Gun}. Il existe deux nouvelles commandes possibles pour le Player: \textbf{ShootL} et \textbf{ShootR}. Lorsque le Player possède au moins une munition et tire, les cellules de la ligne (celles à droite ou à gauche du Player suivant la commande) sont scannées jusqu'à tomber sur un garde ou sur un obstacle (une plateforme, une plateforme métallique, une porte, ou un piège). Si le scan s'arrête  sur un garde, alors on change l'état du garde de sorte que son \textbf{IsShot(G)} soit à True.\\

L'Engine s'occupe de gérer le reste. Lorsque que le Player se trouve sur une cellule contenant un pistolet, cinq munitions sont récupérées, et s'il tire le nombre de munitions est décrémenté. L'opérateur \textrm{NumberBullets : [Engine] $\rightarrow$ int} permet de connaître le nombre de munitions disponibles. Après l'action du Player, l'Engine exécute l'action des gardes seulement lorsque \textbf{GuardTurn(EG)} est à True sauf pour un cas: si \textbf{IsShot(G)} d'un garde retourne True, alors il exécute son action du garde. Au début de chaque tour de l'Engine, l'état des gardes est changé pour que \textbf{IsShot(G)} de chaque retourne False.\\

Lorsqu'un garde fait son action, si \textbf{IsShot(G)} est True alors le garde retourne à sa cellule initiale tout en laissant le trésor qu'il a ramassé s'il en avait un.

\section{Conclusion}
Lors de ce projet, on a pu implémenter un (petit) système entier en utilisant la méthodologie ``programmation par contrats'' et le Model-Based-Testing, et on y a constaté deux choses: premièrement, c'est évidemment un travail très long et fastidieux; en particulier l'implémentation des contrats demande d'écrire beaucoup de code (qu'on pourrait éventuellement générer automatiquement ?). De plus, il est toujours possible de se tromper en traduisant la spécification en contrat.\\


De plus, la non exhaustivité des tests (en particulier dans les case des opérateurs \textrm{Step} de \textrm{Guard}, \textrm{Player} et \textrm{Engine}, ou les cas différents sont beaucoup trop nombreux) nous empêche même d'être sur de l'exact validité du programme: il est toujours possible qu'un cas vraiment spécifique se soit glissé entre les mailles du filet.\\

Il est donc difficile de conclure sur l'efficacité de ces méthodes: même si il est vrai qu'elles permettent d'éviter ``99\%" des bugs, et sont donc plus que suffisantes pour un jeu ou une application sans conséquence, mais probablement pas suffisante pour des applications plus critiques. On a donc hâte de pouvoir explorer des méthodes plus formelles, telle que la preuve de programme, qui donnent plus de certitudes !

\end{document}
